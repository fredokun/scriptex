%%% >>>>>>>>>>>>>>>>>>>>>>>>>>>>>>>>
%%% Tango latex preamble starts here

\documentclass{article}

%%% Tango only supports extended utf-8 encodings
\usepackage{ucs}
\usepackage[utf8x]{inputenc}

%%% <<<<<<<<<<<<<<<<<<<<<<<<<<<<<<<
%%% Tango latex preamble stops here

%%% >>>>>>>>>>>>>>>>>>>>>>>>>>>>>>>>
%%% Tango: Beginning of top-level document
%%% Tango: File = 'basic.tango.tex'
\begin{document}









%%%
%%% Tango Line: 10
\begin{abstract}
%%%
%%% Tango Line: 11
This document is a basic example for Tango.
%%%
%%% Tango Line: 12
Is it not really interesting otherwise.
%%%
%%% Tango Line: 13
\end{abstract}

%%%
%%% Tango Line: 15
\section{About Tango}

%%%
%%% Tango Line: 17
\emph[{'emph_type': '*'}]{Tango} is a document processing system similar to \verb|Sphinx| or Pandoc with
%%%
%%% Tango Line: 18
some 
%%%
%%% Tango Line: 19
specificities. 

%%%
%%% Tango Line: 21
\begin{itemize}
%%%
%%% Tango Line: 22
\item its input format is a simplified variant of LateX
%%%
%%% Tango Line: 23
\item it is primarily aimed at two output formats:
%%%
%%% Tango Line: 24
\begin{itemize}
%%%
%%% Tango Line: 25
\item a latex output for producing PDF files (for screen or printout)
%%%
%%% Tango Line: 26
\item a html/css/javascript format for interactive web pages
%%%
%%% Tango Line: 27
\end{itemize}
%%%
%%% Tango Line: 28
\item its main goal is for producing scientific and technical document with
 %%%
%%% Tango Line: 29
enhanced interactivity without sacrifying the availability of a high-quality printable static version.
%%%
%%% Tango Line: 30
\end{itemize}

%%%
%%% Tango Line: 32
For very pragmatic reasons \emph[{'emph_type': '*'}]{Tango} is written in \url[Python]{http://www.python.org}.

%%%
%%% Tango Line: 34
\section{Conclusion}

%%%
%%% Tango Line: 36
You should \strong[{'strong_type': '*'}]{definitely} try to use \emph{Tango} when it's ready.

%%%
%%% Tango Line: 38
\section{Byebye}

%%%
%%% Tango Line: 40
The end.

%%% <<<<<<<<<<<<<<<<<<<<<<<<<<<<<<<
%%% Tango: End of top-level document
\end{document}
