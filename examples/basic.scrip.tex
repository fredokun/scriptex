
\title{A basic ScripTex example}

\author{Frédéric Peschanski}

\date{\Today}

\copyright[licence=cc-by,year=2013,owner=Frédéric Peschanski]

\begin{abstract}
This document is a basic example for ScripTex.
Is it not really interesting otherwise.
\end{abstract}

\section{About ScripTex}

_ScripTex_ is a document processing system similar to Sphinx or Pandoc with
some % ScripTex will be  by far the best but this is a totally gratuitous comment.
specificities. 

\begin{itemize}
\item its input format is a simplified variant of LateX
\item it is primarily aimed at two output formats:
\begin{itemize}
\item a latex output for producing PDF files (for screen or display)
\item a html/css/javascript format for interactive web pages
\end{itemize}
\item its main goal is for producing scientific and technical document with
 enhanced interactivity without sacrifying the availability of a high-quality printable static version.
\end{itemize}

For very pragmatic reasons _ScripTex_ is written in \url[Python]{http://www.python.org}.

\section{Conclusion}

You should __definitely__ try to use \emph{ScripTex} when it's ready.

\section{Byebye}

The end.