\title{A basic Tango example}

\author{Frédéric Peschanski}

\date{\Today}

%\copyright[licence=cc-by,year=2013,owner=Frédéric Peschanski]

\begin{abstract}
This document is a basic example for Tango.
Is it not really interesting otherwise.
\end{abstract}

\section{About Tango}

*Tango* is a document processing system similar to `Sphinx` or Pandoc with
some % Tango will be  by far the best but this is a totally gratuitous comment.
specificities. 

\begin{itemize}
\item its input format is a simplified variant of LateX
\item it is primarily aimed at two output formats:
\begin{itemize}
\item a latex output for producing PDF files (for screen or printout)
\item a html/css/javascript format for interactive web pages
\end{itemize}
\item its main goal is for producing scientific and technical document with
 enhanced interactivity without sacrifying the availability of a high-quality printable static version.
\end{itemize}

For very pragmatic reasons *Tango* is written in \url[Python]{http://www.python.org}.

\section{Conclusion}

You should **definitely** try to use \emph{Tango} when it's ready.

\section{Byebye}

The end.
